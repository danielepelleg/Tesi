\chapter*{Introduzione}
\label{chap:introduction}
In questa Tesi di Laurea vengono descritte le fasi riguardanti il processo di reingegnerizzazione di un software per la prenotazione di servizi sanitari: Zerocoda. La necessità di questo processo nasce dall'emergenza sanitaria verificatasi a causa della pandemia dell'ultimo anno. Il COVID19 ha stravolto le abitudini di tutti, ma più di ogni altra cosa ha cambiato le interazioni tra le persone. Questo ha portato software come Zerocoda ad un incremento non indifferente delle richieste, e ad un numero sempre maggiore di acquirenti. Questi applicativi esistevano già da tempo, ma prima di oggi sono sempre stati lasciati in disparte, sfavoriti dalla precedenza data ad altri tipi di investimento. Con la pandemia, il loro utilizzo è diventato \emph{una necessità per le aziende sanitarie}, che in questo modo aggiungono alle loro strutture un primo livello di sicurezza nelle interazioni sociali.

\section*{Zerocoda}
Zerocoda è un applicazione sviluppata da Artexe, ora acquistata dall'azienda Maps Group S.p.A. La Tesi presentata è un elaborato di tirocinio svolto presso questa azienda, totalmente in loco presso la sede centrale di Parma. Il fine dell'applicazione è quello di gestire la coda nelle strutture sanitarie. L'applicazione, prima del refactoring, non era ottimizzata per il suo funzionamento. L'architettura monolitica da cui era caratterizzata, unita alle tecnologie obsolete e all'assenza di un vero e proprio layer di API, potevano essere causa di malfunzionamenti o attacchi informatici. I primi obiettivi del refactoring mirano pertanto a risolvere queste sue vulnerabilità. La nuova architettura pensata per l'applicazione fa uso di microservizi, e con questi viene introdotto un layer di REST API.  Sulla base di Zerocoda, è stata recentemente sviluppata una seconda applicazione: \emph{Zerocoda Retail}. Questa seconda applicazione si discosta dalla precedente per il tipo di pubblico a cui essa è rivolta. Zerocoda Retail è stata pensata per quelle aziende che vendono prodotti direttamente al consumatore finale, e che quindi, come i sanitari, hanno necessità di gestire la propria clientela. Zerocoda Retail è stata sviluppata partendo proprio dal codice sorgente di Zerocoda, come suo \emph{fork}. Nonostante le due applicazioni appartengano a realtà molto diverse, queste sono destinate a far parte di un unico grande progetto. Il processo di refactoring presentato nelle prossime pagine stravolge l'architettura del sistema, introducendo nuove tecnologie e conferendo al software \emph{scalabilità}, proprio per permettere alle due, un giorno, di unirsi nella \emph{stessa applicazione Zerocoda}.

\section*{Suddivisione in Capitoli}
L'elaborato si articola in 4 diversi capitoli, i cui contenuti vengono divisi come segue:
\begin{itemize}
    \item \textbf{Stato dell'Arte} vengono presentate le principali tecnologie utilizzate dal sistema, prima e dopo il refactoring. In questo capitolo, inoltre, le nuove tecnologie vengono giustificate nel loro utilizzo, spiegando il perchè rappresentino la scelta giusta per il software creato, con cenni alle alternative possibili.
    \item \textbf{Architettura Funzionale del Sistema} viene descritto il funzionamento dell'applicazione mediante diagrammi UML per facilitarne la comprensione. Si procede ad un analisi più specifica delle tecnologie che la vecchia applicazione utilizza e ne viene illustrata l'architettura. Una volta chiariti i limiti che presenta e la necessità di questo refactoring, viene descritto il modello di sviluppo seguito in fase di progettazione. Infine viene illustrata la nuova architettura software progettata, con riferimenti al nuovo scenario di utilizzo del software e al ruolo delle nuove tecnologie introdotte.
    \item \textbf{Implementazione} all'inizio del capitolo viene fatto riferimento ai design pattern seguiti in fase di sviluppo del software. In seguito, viene percorso per intero il processo implementativo che porta alla creazione delle nuove API REST. Viene descritta la nuova struttura del progetto e come i diversi moduli del sistema cooperano tra loro.
    \item \textbf{Attività di Test} in questo capitolo vengono testate le nuove API implementate con quelle del sistema precedente. I test vengono svolti simulando un carico di utenza sul sistema. I risultati ottenuti vengono descritti mediante l'utilizzo di grafici e tabelle. Infine, i dati appartenenti ai due sistemi sono soggetti a un cofronto, nel quale viene spiegata la motivazione dietro ai risultati ottenuti.
\end{itemize} 