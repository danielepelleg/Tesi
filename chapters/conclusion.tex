\chapter*{Conclusioni}
\label{chap:conclusion}
Durante lo sviluppo dell'elaborato sono stati affrontati diversi temi. Nello Stato dell'Arte (Capitolo \ref{chap:stateofart}) sono state presentate le tecnologie utilizzate dai moduli che compongono i due sistemi. Queste sono state confrontate, per studiare quale fosse la soluzione migliore per il nuovo obiettivo. Lo studio, presentato nel Capitolo \ref{chap:systemdesign} si è rivelato necessario per conoscere in particolare quali fossero i \emph{rischi} derivanti dall'adozione di nuove tecnologie, e per capire se convenisse cambiare quelle del sistema precedente nella sua totalità o solo in parte. Sulla base di questi ragionamenti è stata sviluppata la nuova architettura del sistema. Il tutto è stato fatto tenendo conto di una certezza: il sistema implementato sarebbe dovuto coesistere, per le prime fasi del refactoring, con la \emph{relativa parte legacy}. Nell'architettura del sistema implementato, pertanto è rimasto il `monolite' che adempie a diversi compiti e che rappresenta sostanzialmente il backend di partenza. Questo backend, assieme alle relative API, rimane in appoggio al backoffice, l'interfaccia amministrativa attraverso la quale è possibile aggiungere, per ogni servizio, nuovi appuntamenti prenotabili dall'utente. I processi implementativi dietro la realizzazione della nuova architettura sono stati descritti nel Capitolo \ref{chap:implementation}. Qui sono stati presentati i design pattern seguiti durante lo sviluppo del software e le tecnologie utilizzate, mostrando il funzionamento di una chiamata e il modo in cui i vari componenti del sistema comunicano tra loro. Grazie ai test effettuati per valutare le prestazioni del sistema nel Capitolo \ref{chap:testing} si è constatato che il sistema è pronto ad esssere trasferito sul \emph{branch di produzione}. Il nuovo software è in grado di gestire un carico di utenze maggiore rispetto al precedente, e con tempi di risposta migliori. Il sogno di poter vedere Zerocoda, un'applicazione da sempre caratterizzata da un'architettura monolitica, lavorare attraverso microservizi è ora diventato realtà. L'applicazione ha così raggiunto quella \emph{scalabilità} tanto perseguita in corso di sviluppo, quanto assente in partenza. La Tesi ha quindi raggiunto il suo scopo, in quanto la reingegnerizzazione del software ha portato all'ottenimento di un sistema più performante.

\section{Sviluppi Futuri}
Il prossimo step consisterà quindi nell'installazione del software. Il layer di API implementato verrà spostato sul branch di produzione, in appoggio al Gateway e all'Authentication Server. Il primo utilizzato per il controllo delle chiavi API, mentre il secondo per la gestione dei dati utente in maniera autonoma, per ciascun servizio, dal database. Il miglioramento ottenuto in termini di prestazioni, tuttavia risulterà davvero palpabile solo quando anche \emph{le ultime parti dipendenti dal monolite si appoggeranno al reverse proxy davanti al layer di API}. Questo segnerà il vero completamento dell'azione di refactoring. Per far sì che ciò accada ci sono ancora due fasi da seguire. Il frontend, rimasto inalterato durante lo sviluppo di questo elaborato, dovrà essere soggetto a modifiche per rispondere alle esigenze grafiche ora dettate dalle nuove API. Il database, d'altra parte, dovrà anch'esso subire dei cambiamenti. Le query delle nuove API sono state riscritte, tuttavia è il sistema di salvataggio dei dati che sul database che necessita un cambiamento. Oltre a subire una completa ristrutturazione, per il database verranno studiati i vincoli di integrità migliori, atti a garantirne il corretto funzionamento nel tempo. Al completamento di questa ultima fase il sistema risulterà pronto all'integrazione finale con Zerocoda Royalty. 

\section{Considerazioni}
La pandemia dell'ultimo anno ha contribuito alla messa in luce dei limiti di software come questo. I tagli alla sanità non hanno colpito solo le risorse ospedaliere, ma anche software che come Zerocoda propongono piccoli miglioramenti a degli scenari di vita quotidiana, per le aziende e per le persone. Il mancato impiego di risorse in applicazioni di questo tipo ha fatto sì che le loro tecnologie utilizzate rimanessero indietro rispetto a quelle di altri settori. Solo negli ultimi mesi lo sviluppo di questa applicazioni sono state favorite dalla situazione, che ha permesso loro di avere la giusta importanza nello scenario attuale. Il mancato investimento nello sviluppo di software come questo può portare a una risoluzione dei suoi problemi mediante dei \emph{work around} piuttosto che all'effettivo studio di una soluzione. È evidente che la tecnologia ha sempre più un ruolo chiave nella nostra società, sia per i privati che per le aziende. Investire in questo settore può portare a un'efficienza dei servizi mai vista prima, permettendo alle stesse strutture di essere preparati a casi-limite come quello dell'ultimo anno. Due sono le qualità chiave da perseguire nello sviluppo software: \emph{efficienza} ed \emph{scalabilità}. Il sistema deve sì essere manutenibile nel tempo, ma prima deve essere facilmente aggiornabile all'introduzione di nuove tecnologie, in quanto:
\begin{center}
    \textit{Il software è come l'entropia. É difficile da afferrare, non pesa nulla, e obbedisce alla seconda legge della termodinamica: aumenta sempre.}
\end{center}